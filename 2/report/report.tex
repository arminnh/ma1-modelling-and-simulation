\documentclass[11pt, a4paper, titlepage, openright]{article}

\usepackage{amsmath}
\usepackage[font=small,labelfont=bf]{caption}
\usepackage{float}
\restylefloat{figure}
\usepackage{graphicx}
\usepackage{hyperref}
\usepackage{mathtools}
\usepackage[titletoc, title]{appendix}
\usepackage{listings}
\usepackage{color}
\usepackage{fixltx2e}
\usepackage[bottom]{footmisc}

\usepackage[a4paper, total={6.5in, 9.5in}]{geometry}

\usepackage{eurosym}
\usepackage{graphicx}
\usepackage{wrapfig}
\usepackage{lscape}
\usepackage{rotating}
\usepackage{epstopdf}
\usepackage{amssymb}

\definecolor{dkgreen}{rgb}{0,0.6,0}
\definecolor{gray}{rgb}{0.5,0.5,0.5}
\definecolor{mauve}{rgb}{0.58,0,0.82}

\lstset{frame=tb,
  aboveskip=3mm,
  belowskip=3mm,
  showstringspaces=false,
  columns=flexible,
  basicstyle={\footnotesize\ttfamily},
  numbers=none,
  numberstyle=\tiny\color{gray},
  keywordstyle=\color{blue},
  commentstyle=\color{dkgreen},
  stringstyle=\color{mauve},
  breaklines=true,
  breakatwhitespace=true,
  tabsize=3,
  showstringspaces=false,
  keepspaces=true,
  columns=flexible
  }

\newenvironment{verbquote}
	{\catcode `^=12% Math superscript
	\catcode `_=12% Math subscript
	\catcode `$=12% Math deliniation
	\begin{quote}}
	{\end{quote}}

\begin{document}
\input{title_page.tex}
\tableofcontents
\newpage

\noindent In dit verslag worden oplossingen gegeven voor de opdrachten in het practicum. 
De code voor alle opdrachten staat onder appendix A.

\section{Een aanbevelingssysteem voor films}
	\subsection{Matrixvervollediging}
	
	\subsubsection{Opdracht 2}		
    Om de volle beoordelingenmatrix \textbf{full(R)} (met \( R \in \mathbb{R}^{m \times n} \)) voor te stellen 
    is er \( 8 \times m \times n \) bytes geheugenruimte nodig. \\

    Om een matrix in coordinaatformaat op te slaan zijn er verzamelingen van rij-indices \(i\),
    kolom-indices \(j\), en waarden \( r_{i,j} \) nodig. De verzamelingen van indices kunnen bestaan
    uit gehele getallen. Er is dus \( 4 \times r + 4 \times r + 8 \times r = 16 \times r \) bytes
    geheugenruimte nodig voor het coordinaatformaat, waarbij r het aantal niet-nulwaarden is van \( R\). \\

    Om de rang-r lagerangbenadering 
    \[ R \approx WF^T = w_1 f_1^T + w_2 f_2^T + ... + w_r f_r^T \]
    voor te stellen is er \( 8 \times m \times r + 8 \times n \times r = 8r (m + n) \) bytes geheugenruimte nodig.

	\subsubsection{Opdracht 3}

	\subsubsection{Opdracht 4}

	\subsubsection{Opdracht 5}

	\subsubsection{Opdracht 6}

	\subsubsection{Opdracht 7}

	\subsubsection{Opdracht 8}

	\subsubsection{Opdracht 9}

	\subsubsection{Opdracht 10}

	\subsubsection{Opdracht 11}

	\subsubsection{Opdracht 12}

	\subsubsection{Opdracht 13}

	\subsubsection{Opdracht 14}

	\subsubsection{Opdracht 15}


	\subsection{Clustering}

	\subsubsection{Opdracht 16}

	\subsubsection{Opdracht 17}

	\subsubsection{Opdracht 18}

	\subsubsection{Opdracht 19}
		
		
\section{Evaluatie}
	
	\subsection{Opdracht 21}
		\begin{quote}
			Hoeveel tijd heb je gespendeerd aan het oplossen van de opdrachten? 
			Hoeveel tijd heb je gespendeerd aan het schrijven van het verslag?
		\end{quote}
		\noindent 
	
	\subsection{Opdracht 22}
		\begin{quote}
			In de loop van deze opdracht hebben we allerhande veronderstellingen gemaakt om ons
nieuw aanbevelingssysteem op te stellen. Wat zijn je bedenkingen hierbij? Vind je de resultaten realis-
tisch? Zou je het ontwikkelde aanbevelingssysteem durven toevoegen aan de lijst van aanbevelingssyste-
men van MovieLens?
		\end{quote}
		
		
	
	\subsection{Opdracht 23}
		\begin{quote}
			Welke bedenkingen heb je bij dit practicum? Was de opdracht (veel) te gemakkelijk, (veel) te moeilijk of 
			van een gepaste moeilijkheidsgraad? Wat zou je zelf anders aangepakt hebben?  Was de terminologie voldoende duidelijk?
		\end{quote}
		
		
	

\onecolumn
\appendix
\appendixpage
\addappheadtotoc

\section{Code}
	\subsection{Opdrachten}
		\subsubsection{Opdracht 2}
			%\lstinputlisting[basicstyle=\scriptsize]{../r0679689_estimateParameters.m}
		\bigskip
		
	\subsection{Extra Functies}
		\subsubsection{r0679689\_blabla}
			%\lstinputlisting[basicstyle=\scriptsize]{../r0679689_brownianMove.m}
		\bigskip
		
\end{document}



\iffalse
	We are given the function
	\[
		\begin{aligned}
		f(x) = 1 + T_6(x)    && -1 \le x \le 1
		\end{aligned}
	\]
	
	where \(T_6(x)\) is the Chebyshev polynomial of degree 6.
	We are tasked to calculate the exact integral \[ I = \int_{-1}^1 \! f(x) \, \mathrm{d}x \] using Maple.
	Also, we are tasked to calculate following numerical approximations using Matlab for up to 2 significant numbers:
	\begin{itemize}
		\item The composite trapezoid rule for \( I \)
		\item \( I \) via a Gauss-Legendre integration rule
		\item Monte Carlo integration of  \( I \)
	\end{itemize}
	
	\begin{figure}[H]
		\begin{minipage}[b]{0.49\textwidth}
			% \includegraphics[width=1.0\textwidth]{../maple/mapleChebyPlot.png}
		\end{minipage}
		\hfill
		\begin{minipage}[b]{0.49\textwidth}
			\end{minipage}
		\caption{Two plots of the given function.}
		\label{fig:function}
	\end{figure}
	
	\begin{lstlisting}
		f := x -> 1 + ChebyshevT(6, x);
		int(f(x), x = -1 .. 1);
	\end{lstlisting}
\fi